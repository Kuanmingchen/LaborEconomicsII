\documentclass[12pt]{article}

%%%%%%%%%%%%%%%%%%%%%%%%%%%%%%%%%%%%%%%%%%%%%%%%%%%%%%%%%%%%%%%%%%
%%%Comment out to use biblatex instead of bibtex
%%%%%%%%%%%%%%%%%%%%%%%%%%%%%%%%%%%%%%%%%%%%%%%%%%%%%%%%%%%%%%%%%%
\def\UseBibLatex{1}

%%%%%%%%%%%%%%%%%%%%%%%%%%%%%%%%%%%%%%%%%%%%%%%%%%%%%%%%%%%%%%%%%
%Put all your private style files / class style files in 
%the styles / subdirectory. The following command 
%guarantee that latex would find it.
%%%%%%%%%%%%%%%%%%%%%%%%%%%%%%%%%%%%%%%%%%%%%%%%%%%%%%%%%%%%%%%%%

\makeatletter
\def\input@path{{styles/}}
\makeatother


%%%%%%%%%%%%%%%%%%%%%%%%%%%%%%%%%%%%%%%%%%%%%%%%%%%%%%%%%%%%%%%%%%
%A modified usepackge command that checks for style files in 
%the styles / subdirectory.
%%%%%%%%%%%%%%%%%%%%%%%%%%%%%%%%%%%%%%%%%%%%%%%%%%%%%%%%%%%%%%%%%%
\newcommand{\UsePackage}[1]{%
\IfFileExists{styles/#1.sty}{%
\usepackage{styles/#1}%
}{%
\IfFileExists{../styles/#1.sty}{%
\usepackage{../styles/#1}%
}{%
\usepackage{#1}%
}%
}%
}

\usepackage{palatino}
\usepackage[T1]{fontenc}
\usepackage{mathpazo}
\usepackage{textcomp}

\usepackage{setspace}
\usepackage{amsmath}%
\usepackage{amssymb}%
\usepackage{mathtools}
\usepackage{dsfont}
\usepackage[table]{xcolor}%

\setlength{\marginparwidth}{6cm}
\usepackage{todonotes}
\usepackage[in]{fullpage}%
%\usepackage{enumitem}

\usepackage[amsmath,thmmarks]{ntheorem}
\theoremseparator{.}



\usepackage{titlesec}%
\titlelabel{\thetitle.\hspace{0.5em}}%
\usepackage{xcolor}%
\usepackage{mleftright}%
\usepackage{xspace}%
\usepackage{graphicx}
\usepackage{hyperref}%
\usepackage{etoolbox}
\usepackage{lipsum}
\usepackage{appendix}
\usepackage{tikz}
\usepackage{mathrsfs}
\usepackage{comment}
\usepackage{systeme}




\newcommand{\hrefb}[3][black]{\href{#2}{\color{#1}{#3}}}%
\newcommand{\fakesubsection}[1]{%
  \par\refstepcounter{subsection}% Increase subsection counter
  \subsectionmark{#1}% Add subsection mark (header)
  \addcontentsline{toc}{subsection}{\protect\numberline{\thesubsection}#1}% Add subsection to ToC
  % Add more content here, if needed.
}

\usepackage{tocloft}
\renewcommand{\cftsecfont}{\bfseries}
\renewcommand{\cftsubsecfont}{\bfseries}
\renewcommand{\cftsecpagefont}{\bfseries}
\renewcommand{\cftsubsecpagefont}{\bfseries}

\setlength{\cftbeforesecskip}{0.5em} % Adjust the space before each section entry
\setlength{\cftbeforepartskip}{2em}  % Adjust the space before each part entry

\usepackage{hyperref}%
\hypersetup{%
unicode,
breaklinks,%
colorlinks=true,%
urlcolor=blue,%
linkcolor=[rgb]{0.5,0.0,0.0},%
citecolor=[rgb]{0,0.2,0.445},%
filecolor=[rgb]{0,0,0.4},
anchorcolor=[rgb]={0.0,0.1,0.2}%
}
\usepackage[ocgcolorlinks]{ocgx2}


%%%%%%%%%%%%%%%%%%%%%%%%%%%%%%%%%%%%%%%%%%%%%%%%%%%%%%%%%%%%%%%%%%%%%%%%
%Defining theorem like environments
%



\theoremstyle{break}%
\newtheorem{theorem}{Theorem}[section]
\theoremstyle{break}%
\newtheorem{lemma}[theorem]{Lemma}
\theoremstyle{break}%
\newtheorem{conjecture}[theorem]{Conjecture}
\theoremstyle{break}%
\newtheorem{corollary}[theorem]{Corollary}
\theoremstyle{break}%
\newtheorem{claim}[theorem]{Claim}%
\theoremstyle{break}%
\newtheorem{fact}[theorem]{Fact}
\theoremstyle{break}%
\newtheorem{observation}[theorem]{Observation}
\theoremstyle{break}%
\newtheorem{invariant}[theorem]{Invariant}
\theoremstyle{break}%
\newtheorem{question}[theorem]{Question}
\theoremstyle{break}%
\newtheorem{proposition}[theorem]{Proposition}
\theoremstyle{break}%
\newtheorem{openproblem}[theorem]{OpenProblem}

\theoremstyle{break}%
\newtheorem*{remark}[theorem]{Remark}%
\theoremstyle{break}%
\newtheorem*{definition}{Definition}
\theoremstyle{break}%
\newtheorem{example}[theorem]{Example}
\theoremstyle{break}%
\newtheorem{exercise}{Exercise}[section]
\theoremstyle{break}%
\newtheorem{problem}{Problem}[section]
\theoremstyle{break}%
\newtheorem{xca}[theorem]{Exercise}
\theoremstyle{break}%
\newtheorem{exercise_h}[theorem]{Exercise}
\theoremstyle{break}%
\newtheorem{assumption}{Assumption}[section]%
\theoremstyle{break}%
\newtheorem{axiom*}{Axiom}%

%Proof environment
\newcommand{\myqedsymbol}{\rule{2mm}{2mm}}

\theoremheaderfont{\itshape}%
\theorembodyfont{\upshape}%
\theoremstyle{nonumberplain}%
\theoremseparator{}%
\theoremsymbol{\myqedsymbol}%
\newtheorem{proof}{Proof.}%
\newtheorem{solution}{Solution.}
\newenvironment{pf}{%
\begin{proof}$ $\par\nobreak\ignorespaces
}{%
\end{proof}
}

\newenvironment{sol}[1][]
  {\begin{center}
  \begin{tabular}{|p{\textwidth}|}
  \hline\\[-1em]
  \textit{Solution. #1.}\\
  }
  { 
  \\[0.5em]\hline
  \end{tabular} 
  \end{center}
  }



\newtheorem{proofof}{Proofof\!}%

%theorem block end
%%%%%%%%%%%%%%%%%%%%%%%%%%%%%%%%%%%%%%%%%%%%%%%%%%%%%%%%%%%%%%%%%%%%


%%%%%%%%%%%%%%%%%%%%%%%%%%%%%%%%%%%%%%%%%%%%%%%%%%%%%%%%%%%%%%%%%%5
%Color emph

\providecommand{\emphind}[1]{}%
\renewcommand{\emphind}[1]{\emph{#1}\index{#1}}

\definecolor{blue25emph}{rgb}{0,0,11}

\providecommand{\emphic}[2]{}
\renewcommand{\emphic}[2]{\textcolor{blue25emph}{%
\textbf{\emph{#1}}}\index{#2}}

\providecommand{\emphi}[1]{}%
\renewcommand{\emphi}[1]{\emphic{#1}{#1}}

\definecolor{almostblack}{rgb}{0,0,0.3}

\providecommand{\emphw}[1]{}%
\renewcommand{\emphw}[1]{{\textcolor{almostblack}{\emph{#1}}}}%

\providecommand{\emphOnly}[1]{}%
\renewcommand{\emphOnly}[1]{\emph{\textcolor{blue25}{\textbf{#1}}}}

%%%%%%%%%%%%%%%%%%%%%%%%%%%%%%%%%%%%%%%%%

%%%%%%%%%%%%%%%%%%%%%%%%%%%%%%%%%%%%%%%%%%%%%%%%%%%%%%%%%%%%%%%%%%%%%%
%Handling references
%%%%%%%%%%%%%%%%%%%%%%%%%%%%%%%%%%%%%%%%%%%%%%%%%%%%%%%%%%%%%%%%%%%%%%

\newcommand{\HLink}[2]{\hyperref[#2]{#1~\ref*{#2}}}
\newcommand{\HLinkSuffix}[3]{\hyperref[#2]{#1\ref*{#2}{#3}}}

\newcommand{\figlab}[1]{\label{fig:#1}}
\newcommand{\figref}[1]{\HLink{Figure}{fig:#1}}

\newcommand{\thmlab}[1]{{\label{theo:#1}}}
\newcommand{\thmref}[1]{\HLink{Theorem}{theo:#1}}

\newcommand{\exelab}[1]{{\label{exe:#1}}}
\newcommand{\exeref}[1]{\HLink{Exercise}{exe:#1}}

\newcommand{\prolab}[1]{{\label{pro:#1}}}
\newcommand{\proref}[1]{\HLink{Problem}{pro:#1}}

\newcommand{\remlab}[1]{\label{rem:#1}}
\newcommand{\remref}[1]{\HLink{Remark}{rem:#1}}%

\newcommand{\corlab}[1]{\label{cor:#1}}
\newcommand{\corref}[1]{\HLink{Corollary}{cor:#1}}%

\providecommand{\deflab}[1]{}
\renewcommand{\deflab}[1]{\label{def:#1}}
\newcommand{\defref}[1]{\HLink{Definition}{def:#1}}

\newcommand{\lemlab}[1]{\label{lemma:#1}}
\newcommand{\lemref}[1]{\HLink{Lemma}{lemma:#1}}%

\providecommand{\eqlab}[1]{}%
\renewcommand{\eqlab}[1]{\label{equation:#1}}
\renewcommand{\eqref}[1]{\HLinkSuffix{Eq.~(}{equation:#1}{)}}

%%%%%%%%%%%%%%%%%%%%%%%%%%%%%%%%%%%%%%%%%%%%%%%%%%%%%%%%%%%%%%%%%%%

\newcommand{\remove}[1]{}%


\renewcommand{\th}{th\xspace}




%%%%%%%%%%%%%%%%%%%%%%%%%%%%%%%%%%%%%%%%%%%%%%%%%%%%%%%%%%%%%%%%%%%%%%%%%
%Defining comptenum environment using enumitem
\usepackage{enumitem}

\newlist{compactenumA}{enumerate}{5}%
\setlist[compactenumA]{topsep=0pt,itemsep=-1ex,partopsep=1ex,parsep=1ex,%
label=(\Alph*)}%

\newlist{compactenuma}{enumerate}{5}%
\setlist[compactenuma]{topsep=0pt,itemsep=-1ex,partopsep=1ex,parsep=1ex,%
label=(\alph*)}%

\newenvironment{thmenum}{%
\begin{enumerate}[label=\small(\alph*\small),topsep=8pt,itemsep=4pt,partopsep=0pt, parsep=0pt]
}{%
\end{enumerate}
}

\newlist{compactenumI}{enumerate}{5}%
\setlist[compactenumI]{topsep=0pt,itemsep=-1ex,partopsep=1ex,parsep=1ex,%
label=(\Roman*)}%

\newlist{compactenumi}{enumerate}{5}%
\setlist[compactenumi]{topsep=0pt,itemsep=-1ex,partopsep=1ex,parsep=1ex,%
label=(\roman*)}%

\newlist{compactitem}{itemize}{5}%
\setlist[compactitem]{topsep=0pt,itemsep=-1ex,partopsep=1ex,parsep=1ex,%
label=\ensuremath{\bullet}}%


%%%%%%%%%%%%%%%%%%%%%%%%%%%%%%%%%%%%%%%%%%%%%%%%%%%%%%%%%%%%%%%%%%%%%%%%%%

%%%%%%%%%%%%%%%%%%%%%%%%%%%%%%%%%%%%%%%%%%%%%%%%%%%%%%%%%%%%%%%%%%%
%%%%%%%%%%%%%%%%%%%%%%%%%%%%%%%%%%%%%%%%%%%%%%%%%%%%%%%%%%%%%%%%%%%

\numberwithin{figure}{section}%
\numberwithin{table}{section}%
\numberwithin{equation}{section}%



%%%%%%%%%%%%%%%%%%%%%%%%%%%%%%%%%%%%%%%%%%%%%%%%%%%%%%%%%%%%%%%%%%%
%%%%%%%%%%%%%%%%%%%%%%%%%%%%%%%%%%%%%%%%%%%%%%%%%%%%%%%%%%%%%%%%%%%
%Papers specific commands...
%%%%%%%%%%%%%%%%%%%%%%%%%%%%%%%%%%%%%%%%%%%%%%%%%%%%%%%%
%%%%%%%%%%%%%%%%%%%%%%%%%%%%%%%%%%%%%%%%%%%%%%%%%%%%%%%%



%%%%%%%%%%%%%%%%%%%%%%%%%%%%%%%%%%%%%%%%%%%%%%%%%%%%%%%%
%%Begin Ipe Preamble
%%%%%%%%%%%%%%%%%%%%%%%%%%%%%%%%%%%%%%%%%%%%%%%%%%%%%%%%


%%%%%%%%%%%%%%%%%%%%%%%%%%%%%%%%%%%%%%%%%%%%%%%%%%%%%%%%
%%End Ipe Preamble
%%%%%%%%%%%%%%%%%%%%%%%%%%%%%%%%%%%%%%%%%%%%%%%%%%%%%%%%
%
\newcommand{\Set}[2]{\left\{ #1 \;\middle\vert\; #2 \right\}}

\newcommand{\pth}[1]{\mleft(#1\mright)}%

\newcommand{\E}{{\mathbb{E}}}
\newcommand{\diam}[1]{\operatorname{diam}\mleft(#1\mright)}
\newcommand{\osc}[2]{\operatorname{osc}\mleft(#1,\, #2\mright)}

\newcommand{\Ex}[1]{\E\mleft[ #1 \mright]}

\newcommand{\ceil}[1]{\mleft\lceil {#1} \mright\rceil}
\newcommand{\floor}[1]{\mleft\lfloor {#1} \mright\rfloor}
\makeatletter
\renewcommand\d[1]{\mspace{6mu}\mathrm{d}#1\@ifnextchar\d{\mspace{-3mu}}{}}
\makeatother
\def\at{
  \left.
  \vphantom{\int}
  \right|
}

\newcommand{\brc}[1]{\left\{ {#1} \right\}}
\newcommand{\sbrc}[1]{\mleft[#1\mright]}
\newcommand{\set}[1]{\brc{#1}}%

\newcommand{\abs}[1]{\left\lvert {#1} \right\rvert}%
\newcommand{\norm}[1]{\left\lVert {#1} \right\rVert}

\newcommand{\ds}{\displaystyle}%

\newcommand{\R}{\mathbb{R}}%
\newcommand{\N}{\mathbb{N}}
\newcommand{\Q}{\mathbb{Q}}
\newcommand{\Z}{\mathbb{Z}}
\newcommand{\C}{\mathbb{C}}
\newcommand{\F}{\mathcal{F}}
\newcommand{\A}{\mathcal{A}}

\newcommand{\exist}{\exists\:}
\newcommand{\forany}{\forall\:}


\title{
    Labor Economics Homework 1 \\
    \large Due: \\
    }

\author{}

\date{}

\begin{document}
\setstretch{1.2}

\maketitle
\vspace{-2cm}

\section{Julia and VSCode}
\begin{enumerate}
    \item Go to \url{https://julialang.org/downloads/} and download the latest stable version of Julia. 
    \item Go to \url{https://code.visualstudio.com/download} and download Visual Studio Code.
    \item Go to the extensions tab in VSCode and install the Julia extension.
\end{enumerate}
Now you should be able to run Julia code in VSCode. To install 
packages, you can open the terminal and type \texttt{julia}; then 
type \texttt{]add PackageName} to install the package 
\texttt{PackageName}. For more details, see 
\url{https://julia.quantecon.org/intro.html}.

\section{Methodology}
Suppose that you are a restaurant owner and you are considering 
increasing the price. You know that the customers' preference 
can be represented by 
\begin{equation*}
    u_i(x, y) = \frac{\epsilon_i}{1-\epsilon_i}\log (x) + \log (y), 
\end{equation*}
where $x$ is the restaurant's food and $y$ represents the other 
goods. $\epsilon_i\overset{iid}{\sim}Beta(\alpha, \beta)$ is an 
unobserved term, and you do not observe $\alpha$ and $\beta$ 
either. Each customer $i$ faces the budget constraint
\begin{equation*}
    p x + y \leq w_i,
\end{equation*}
where the price of the other foods is normalized to 1. To 
simplify the problem, assume that $w_i = 1$ for all $i$.
\begin{enumerate}
    \item Solve for the demand of $x_i$.
    \item After surveying the customers, you obtain the data 
    $\set{x_i}_{i=1}^N$. Write down the maximum likelihood estimator 
    of $\alpha$ and $\beta$.
    \item Suppose that you have estimated $\alpha$ and $\beta$. 
    How much, on average, would the demand of $x_i$ be if you 
    set the price to $p'$?
\end{enumerate}

%\begin{comment}
\begin{sol}[2.1]
    The F.O.C.\ is 
    \begin{equation*}
        \left\{
        \begin{aligned}
            \frac{\epsilon_i}{1-\epsilon_i}\frac{1}{x_i} &= \lambda p, \\
            \frac{1}{y_i} &= \lambda, \\
            p x_i + y_i &= 1.
        \end{aligned}
        \right .
    \end{equation*}
    Solving the system of equations yields 
    \begin{equation*}
        x_i = \frac{\epsilon_i}{p}.
    \end{equation*}
    This is the demand for the customer $i$.
    \solend
\end{sol}
%\end{comment}

%\begin{comment}
\begin{sol}[2.2]
    Let $f(\cdot;\alpha, \beta)$ be the density of $Beta(\alpha, \beta)$. 
    The likelihood function is 
    \begin{equation*}
        L(\alpha, \beta) = \prod_{i=1}^N f(\epsilon_i;\alpha, \beta) = \prod_{i=1}^{N} f(p x_i;\alpha, \beta).
    \end{equation*}
    The maximum likelihood estimator is 
    \begin{equation*}
        (\hat{\alpha}, \hat{\beta}) = \argmax_{\alpha, \beta} L(\alpha, \beta).
    \end{equation*}
\end{sol}
%\end{comment}

%\begin{comment}
\begin{sol}[2.3]
    On average, the demand of $x_i$ would be
    \begin{equation*}
        \Ex{x_i} = \Ex{\frac{\epsilon_i}{p'}} = \frac{1}{p'}\frac{\alpha}{\alpha + \beta}.
    \end{equation*}
    \solend
\end{sol}
%\end{comment}



\section{Robinson Crusoe's Coconuts}
Robinson Crusoe finds $z_0$ units of coconut on the beach in 
the morning of day $0$. Each day, he can either consume the 
coconuts or leave them for the next day. If he consumes $c_t$ 
coconuts on day $t$, he will have $z_{t+1} = z_t - c_t$. In 
each day, the utility of consuming $c_t$ coconuts is $u(c_t) = 
\sqrt{c_t}$. Furthermore, the coconuts will rot at the end of 
day $T$. Given the time discount factor $\beta\in(0,1)$, 
Crusoe's prblem is thus formulated as follows: 
\begin{equation*}
    \max_{\{c_t\}_{t=0}^T} \sum_{t=0}^T \beta^t u(c_t) 
    \quad \text{s.t.} \quad z_{t+1} = z_t - c_t. 
\end{equation*}
\begin{enumerate}
    \item Write down the value function $v(z, t)$ and the 
    Bellman equation. 
    \item Write down the first-order condition of the problem. 
    \item Solve the optimal consumption $c^*_0$ corresponding 
    to $z_0$ analytically. 
    \item Now instead of the coconuts, Crusoe finds $z_0$ units 
    of canned food. Since the canned food will not rot, the 
    problem is formulated as 
    \begin{equation*}
        \max_{\{c_t\}_{t=0}^\infty} \sum_{t=0}^\infty \beta^t u(c_t) 
        \quad \text{s.t.} \quad z_{t+1} = z_t - c_t.
    \end{equation*}
    Write down the value function and the Bellman equation. 
    Solve the optimal consumption $c^*_0$ corresponding to 
    $z_0$ analytically.
\end{enumerate}

%\begin{comment}
\begin{sol}[3.1]
    The value function is 
    \begin{equation*}
        v(z, t) = \max_{\set{c_s}_{s=t}^T} \sum_{s=t}^T \beta^s u(c_s) 
    \end{equation*}
    with $z = \sum_{s=t}^{T}c_s$. The Bellman equation is 
    \begin{equation*}
        v(z, t) = \max_{c_{t}} \brc{u(c_t) + \beta v(z-c_t, t+1)}.
    \end{equation*}
    \solend
\end{sol}
%\end{comment}

%\begin{comment}
\begin{sol}[3.2]
    The F.O.C. is 
    \begin{equation*}
        u'(c_t) = \beta u'(c_{t+1})\quad \forany t = 0,1,\ldots,T-1.
    \end{equation*}
    Since $u(c_t) = \sqrt{c_t}$, the F.O.C. becomes 
    \begin{equation*}
        \frac{1}{2\sqrt{c_t}} = \beta \frac{1}{2\sqrt{c_{t+1}}}.
    \end{equation*}
    \solend
\end{sol}
%\end{comment}

%\begin{comment}
\begin{sol}[3.3]
    Rearranging the F.O.C.\ yields 
    \begin{equation*}
        c_{t+1} = \beta^2 c_t.
    \end{equation*}
    Combining this with the budget constraint $z_0 = 
    \sum_{s=0}^{T}c_s$, we have 
    \begin{equation*}
        z_0 = c_0 + \beta^2 c_0 + \cdots + \beta^{2T}c_0 
        = c_0 \sum_{s=0}^{T} \beta^{2s} = c_0 \frac{1-\beta^{2(T+1)}}{1-\beta^2}.
    \end{equation*}
    Thus 
    \begin{equation*}
        c^*_0 = \frac{z_0(1-\beta^2)}{1-\beta^{2(T+1)}}.
    \end{equation*}
    \solend
\end{sol}
%\end{comment}

%\begin{comment}
\begin{sol}[3.4]
    Taking the limit as $T\to\infty$, the value function is 
    \begin{equation*}
        v(z) = \max_{\set{c_s}_{s=t}^\infty} \sum_{s=t}^\infty \beta^s u(c_s). 
    \end{equation*}
    Observe that the value function doe not depend on $t$ since 
    today's problem is the same as tomorrow's. \\
    The Bellman equation becomes
    \begin{equation*}
        v(z) = \max_{c} \brc{u(c) + \beta v(z-c)}.
    \end{equation*}
    The F.O.C. is the same as \textit{3.3}. Thus the optimal 
    consumption is 
    \begin{equation*}
        c^*_0 = z_0(1-\beta^2)
    \end{equation*}
    by taking $T\to\infty$.
    \solend
\end{sol}
%\end{comment}

\section{McCall's Job Search Model}
A worker is finding a job. In each period, the worker receives 
a job offer with wage $w_t$. The wage is i.i.d. across periods 
in $Beta(\alpha,\beta)$. The worker can either 
accept the offer ($s_t=1$) and work at the wage $w_t$ forever 
or reject it ($s_t=0$) and receives $\underbar{c}$ as the 
unemployment benefit. Once the worker accepts an offer, he 
cannot search for another job or quit. Assume that the worker 
cannot borrow or save; thus $u(c_t)=w_t$. 
With the discount factor $\rho\in(0,1)$, the worker's problem 
is 
\begin{equation*}
    \max_{\{s_t\}_{t=0}^\infty} \E_0\sbrc{\sum_{t=0}^\infty \rho^t(w_ts_t+(1-s_t)\underbar{c})}\quad \text{s.t. $\Set{s_t}{s_t\in\set{0,1}}$ is a non-decreasing sequence}.
\end{equation*}
\begin{enumerate}
    \item Write down the Bellman equation.
    \item Define the Bellman operator and verify that it is a 
    contraction mapping on the space of bounded continuous functions 
    defined on $[0,1]$ with the supremum norm.\\
    (Hint: $\abs{\max\brc{a,b}-\max\brc{a,c}}\leq\abs{b-c}$ for real $a,b,c$.)
    \item Given $\rho = 0.95$, $\underbar{c} = 0.1$, $\alpha = 2$, 
    and $\beta = 5$, solve the model numerically.
    \item Plot the value function and the optimal policy function as functions 
    of $w$. What is the reservation wage? 
    \item Simulate the model for 1000 homogeneous agents. Plot the distribution 
    of the time to find a job. 
    \item Plot the reservation wage as a function of $\underbar{c}$ and repeat 
    the simulation in problem 5 for different value of $\underbar{c}$. Explain 
    your results.
\end{enumerate}

%\begin{comment}
\begin{sol}[4.1]
    The Bellman equation is 
    \begin{equation*}
        v(w) = \max \brc{\frac{w}{1-\rho}, \underbar{c} + \rho \int v(w') F(dw')},
    \end{equation*}
    where $F$ is the CDF of $Beta(\alpha,\beta)$.
    \solend
\end{sol}
%\end{comment}

%\begin{comment}
\begin{sol}[4.2]
    The Bellman operator is 
    \begin{equation*}
        (Tv)(w) = \max \brc{\frac{w}{1-\rho}, \underbar{c} + \rho \int v(w') F(dw')}.
    \end{equation*}
    Now let $v_1$ and $v_2$ be two bounded continuous functions on $[0,1]$ and 
    $\norm{\cdot}$ denote the supremum norm. Then
    \begin{equation*}
        \begin{aligned}
            \norm{Tv_1 - Tv_2} &= \sup_{w\in[0,1]}\Bigg\lvert\max\brc{\frac{w}{1-\rho}, \underbar{c} + \rho \int v_1(w') F(dw')}\\ 
            &\qquad- \max\brc{\frac{w}{1-\rho}, \underbar{c} + \rho \int v_2(w') F(dw')}\Bigg\rvert \\
            &\leq \sup_{w\in[0,1]}\abs{\rho \int v_1(w') F(dw') - \rho \int v_2(w') F(dw')}\\ 
            &\leq \rho \int \sup_{w\in[0,1]}\abs{v_1(w') - v_2(w')} F(dw')\\
            &= \rho \int \norm{v_1 - v_2} F(dw') = \rho \norm{v_1 - v_2}.
        \end{aligned}
    \end{equation*}
    Thus $T$ is a contraction mapping. 
    \solend
\end{sol}
%\end{comment}

%\begin{comment}
\begin{sol}[Alternative Approach for 4.2]
    One may observe that there exists a ``reservation wage'', $\Bar{w}$, 
    where the worker is indifferent between accepting and rejecting the 
    offer. That is,
    \begin{equation*}
        \frac{\Bar{w}}{1-\rho} = \underbar{c} + \rho \int v(w') F(dw').
    \end{equation*} 
    By the Bellman equation, substituting $v(w')$ yields
    \begin{equation*}
        \frac{\Bar{w}}{1-\rho} = \underbar{c} + \rho \int \max\brc{\frac{w'}{1-\rho}, \frac{\Bar{w}}{1-\rho}} F(dw'), 
    \end{equation*}
    which is equivalent to 
    \begin{equation*}
        \Bar{w} = (1-\rho)\underbar{c} + \rho \int \max\brc{w', \Bar{w}} F(dw').
    \end{equation*}
    Define the new operator $U$ defined on $[0,1]$ as 
    \begin{equation*}
        U\Bar{w} = (1-\rho)\underbar{c} + \rho \int \max\brc{w', \Bar{w}} F(dw').
    \end{equation*}
    We claim that $U$ is a contraction mapping. \\
    Let $\Bar{w}_1, \Bar{w}_2\in[0,1]$. Then 
    \begin{equation*}
        \begin{aligned}
            \abs{U\Bar{w}_1 - U\Bar{w}_2} 
            &= \abs{\rho \int \max\brc{w', \Bar{w}_1} F(dw') - \rho \int \max\brc{w', \Bar{w}_2} F(dw')}\\
            &= \rho\abs{\int \max\brc{w', \Bar{w}_1}-\max\brc{w', \Bar{w}_2} F(dw')}\\
            &\leq \rho \int\abs{\max\brc{w', \Bar{w}_1}-\max\brc{w', \Bar{w}_2}} F(dw')\\
            &\leq \rho \int\abs{\Bar{w}_1-\Bar{w}_2} F(dw') = \rho \abs{\Bar{w}_1-\Bar{w}_2}\int F(dw') = \rho \abs{\Bar{w}_1-\Bar{w}_2}.
        \end{aligned}
    \end{equation*}
    Thus $U$ is a contraction mapping.\\ 
    Once we obtain the reservation wage, the decision rule becomes 
    obvious: accept the offer if $w\geq\Bar{w}$ and reject it otherwise. 
    We also implement this approach in the Julia code.  
    \solend
\end{sol}
%\end{comment}

%\begin{comment}
\begin{sol}[4.3, 4.4, 4.5, 4.6]
    See the Julia code file.
    \solend 
\end{sol}
%\end{comment}
\end{document}