\documentclass[12pt]{article}
\input{prefix.tex}%
\input{symbols.tex}

\title{
    Labor Economics Homework 4 \\
    \large Due: \\
    }

\author{}

\date{}

\begin{document}    
\setstretch{1.2}

\maketitle
\vspace{-2cm}

\section{Occpational Choice Model}
Suppose a worker $i$ chooses an occupation $j\in\set{0,1,\ldots,J}$ 
to maximize the utility function 
\begin{equation*}
    u_i(j) = \alpha_j + \epsilon_{ij},
\end{equation*}
where $\epsilon_{ij}\overset{iid}{\sim}T1EV$ across workers and occupations 
that is observed only by the individual but not us. $\alpha_j$ is the 
parameter that we want to estimate. We normalize $\alpha_0 = 0$.
\begin{enumerate}
    \item What is the probability of observing a worker in occupation $j$?
    \item Use the parameters $(\alpha_1, \alpha_2, \alpha_3) = (0.2, 0.3, -0.1)$ 
    to simulate the data for $1000$ workers and $J=3$ occupations.
    \item Derive the likelihood function for the parameters $\alpha_j$.
    \item Estimate the parameters $\alpha_j$ via MLE. Can you recover the true 
    parameters?
    \item Compute $\Pr(D_i = 1)/\Pr(D_i = 2)$ for the cases $J = 2$ and $J = 3$. 
    What do you find?
\end{enumerate}
%\begin{comment}
\begin{sol}[1.1]
    By the $T1EV$ assumption, we have 
    \begin{equation*}
        \Pr(D_i = j) = \Pr(\alpha_j + \epsilon_{ij} > \alpha_k + \epsilon_{ik} \text{ for all } k\neq j) = \frac{\exp(\alpha_j)}{\sum_{k=0}^{J}\exp(\alpha_k)}.
    \end{equation*}
    \solend
\end{sol}
%\end{comment}
%\begin{comment}
\begin{sol}[1.2, 1.4]
    See the Julia code file.
    \solend
\end{sol}
%\end{comment}
%\begin{comment}
\begin{sol}[1.3]
    The likelihood function is
    \begin{equation*}
        L(\alpha_1, \alpha_2, \alpha_3) = \prod_{i=1}^{1000}\prod_{j=0}^{J}\Pr(D_i = j)^{\mathds{1}\brc{D_i = j}} = \prod_{i=1}^{1000}\prod_{j=0}^{J}\brc{\frac{\exp(\alpha_j)}{\sum_{k=0}^{J}\exp(\alpha_k)}}^{\mathds{1}\brc{D_i = j}}.
    \end{equation*}
    \solend
\end{sol}
%\end{comment}
%\begin{comment}
\begin{sol}[1.5]
    By 1.1, for $J = 2$, we have 
    \begin{equation*}
        \frac{\Pr(D_i = 1)}{\Pr(D_i = 2)} = \frac{\exp(\alpha_1)}{\exp(\alpha_2)} = \exp(\alpha_1 - \alpha_2).
    \end{equation*}
    For $J = 3$, we have
    \begin{equation*}
        \frac{\Pr(D_i = 1)}{\Pr(D_i = 2)} = \frac{\exp(\alpha_1)}{\exp(\alpha_2)} = \exp(\alpha_1 - \alpha_2).
    \end{equation*}
    Notice that they are the same. This is called the IIA assumption. 
    One should be careful when using the multinomial logit model despite its 
    convenience.
    \solend
\end{sol}
%\end{comment}

\section{Inventory Dynamics}
Consider a firm has a warehouse to store its inventory with 
a capacity of $S$ units. The firm faces demand $D_t$ in each 
period $t$. In each period, the firm can choose to 
restock ($s_t = 1$) the inventory to $S$ units or not ($s_t = 0$). 
The profit flow is 
\begin{equation*}
    \pi(s_t, I_t, \epsilon_{1t}, \epsilon_{0t}) = 
    \begin{cases*}
        pD_t - cS - h(S - I_t) + \epsilon_{1t} & if $s_t = 1$, \\
        p\min\brc{D_t, I_t} - cI_t + \epsilon_{0t} & if $s_t = 0$,
    \end{cases*}
\end{equation*}  
where $I_t$ is the inventory level at the beginning of period $t$, 
$p$ is the price of the product, $c$ is the cost of holding one unit 
of inventory, $\epsilon_{st}$ are random terms iid across choice and 
time following $T1EV$ distribution. Notice that $\epsilon_{st}$ is 
observed by the firm but not us. $h$ is the cost of restocking with 
\begin{equation*}
    h(x) = \theta_0 + \theta_1x.
\end{equation*}
The law of motion for the inventory level is 
\begin{equation*}
    I_{t+1} = 
    \begin{cases*}
        S & if $s_t = 1$, \\
        \max\brc{0,I_t - D_t} & if $s_t = 0$.
    \end{cases*}
\end{equation*}
\begin{enumerate}
    \item Assume that $D_t\overset{iid}{\sim}BetaBinomial(\alpha, \beta, S)$, $S=20$.
    Generate the transition array $Q$ for the inventory level $I_t$.
    \item Let the discount factor be $\rho\in (0,1)$. Write down the Bellman equation 
    for the firm's problem. 
    \item Given the parameters $(p,c,\theta_0,\theta_1,\alpha,\beta,\rho) = (1.3, 1.0, 3.0, 0.7,5.0,2.0,0.95)$,
    solve the firm's policy function using the value function 
    iteration. 
    \item Given the same parameters, solve the firm's policy function using the policy function 
    iteration. \\
    (Hint: In $T_\sigma$ step, $v_\sigma = (I-\rho Q_\sigma)^{-1}\pi_\sigma$, where $Q_\sigma$ is the transition matrix 
    given the policy and $\pi_\sigma$ is the profit flow given the policy.)
    \item Simulate the data $\set{(\min\brc{D_t, I_t}, I_t, s_t)}_{t=1}^{500}$. Estimate the 
    transition matrix parameters $\alpha$ and $\beta$ for the inventory level $I_t$ 
    using the data. 
    \item Estimate the parameters $(p,c,\theta_0,\theta_1,\rho)$ using NFXP algorithm. 
    \item Estimate the parameters $(p,c,\theta_0,\theta_1,\rho)$ using CCP.
    \item Simulate and plot the inventory level $I_t$ against $t$ for $100$ periods for 
    different price levels $p\in\set{1.1, 1.3, 1.5}$ and storage cost $c\in\set{0.8, 1.0, 1.2}$. 
    Explain your findings.
\end{enumerate}


\end{document}
