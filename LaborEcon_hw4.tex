\documentclass[12pt]{article}
\input{prefix.tex}%
\input{symbols.tex}

\title{
    Labor Economics Homework 4 \\
    \large Due: \\
    }

\author{}

\date{}

\begin{document}    
\setstretch{1.2}

\maketitle
\vspace{-2cm}

\section{Occpational Choice Model}
Suppose a worker $i$ chooses an occupation $j\in\set{0,1,\ldots,J}$ 
to maximize the utility function 
\begin{equation*}
    u_i(j) = \alpha_j + \epsilon_{ij},
\end{equation*}
where $\epsilon_{ij}\overset{iid}{\sim}T1EV$ across workers and occupations 
that is observed only by the individual but not us. $\alpha_j$ is the 
parameter that we want to estimate. We normalize $\alpha_0 = 0$.
\begin{enumerate}
    \item What is the probability of observing a worker in occupation $j$?
    \item Use the parameters $(\alpha_1, \alpha_2, \alpha_3) = (0.2, 0.3, -0.1)$ 
    to simulate the data for $1000$ workers and $J=3$ occupations.
    \item Derive the likelihood function for the parameters $\alpha_j$.
    \item Estimate the parameters $\alpha_j$ via MLE. Can you recover the true 
    parameters?
    \item Use the estimated parameters to simulate $1000$ samples. 
    Drop the occupation $j=3$ and resimulate $1000$ samples. Compare  
    $\frac{\sum_{i=1}^{1000}\mathds{1}\brc{D_i=1}}{\sum_{i=1}^{1000}\mathds{1}\brc{D_i=2}}$ 
    in the two scnarios, where $D_i\in \set{0,\ldots,J}$ is the occupation 
    that worker $i$ chooses.
\end{enumerate}

\section{Bus Engine Replacement Model}


\end{document}
