\documentclass[12pt]{article}
\input{prefix.tex}%
\input{symbols.tex}

\title{
    Labor Economics Homework 4 \\
    \large Due: \\
    }

\author{}

\date{}

\begin{document}    
\setstretch{1.2}

\maketitle
\vspace{-2cm}

\section{Occpational Choice Model}
Suppose a worker $i$ chooses an occupation $j\in\set{0,1,\ldots,J}$ 
to maximize the utility function 
\begin{equation*}
    u_i(j) = \alpha_j + \epsilon_{ij},
\end{equation*}
where $\epsilon_{ij}\overset{iid}{\sim}T1EV$ across workers and occupations 
that is observed only by the individual but not us. $\alpha_j$ is the 
parameter that we want to estimate. We normalize $\alpha_0 = 0$.
\begin{enumerate}
    \item What is the probability of observing a worker in occupation $j$?
    \item Use the parameters $(\alpha_1, \alpha_2, \alpha_3) = (0.2, 0.3, -0.1)$ 
    to simulate the data for $1000$ workers and $J=3$ occupations.
    \item Derive the likelihood function for the parameters $\alpha_j$.
    \item Estimate the parameters $\alpha_j$ via MLE. Can you recover the true 
    parameters?
    \item Compute $\Pr(D_i = 1)/\Pr(D_i = 2)$ for the cases $J = 2$ and $J = 3$. 
    What do you find?
\end{enumerate}
%\begin{comment}
\begin{sol}[1.1]
    By the $\it{T1EV}$ assumption, we have 
    \begin{equation*}
        \Pr(D_i = j) = \Pr(\alpha_j + \epsilon_{ij} > \alpha_k + \epsilon_{ik} \text{ for all } k\neq j) = \frac{\exp(\alpha_j)}{\sum_{k=0}^{J}\exp(\alpha_k)}.
    \end{equation*}
    \solend
\end{sol}
%\end{comment}
%\begin{comment}
\begin{sol}[1.2, 1.4]
    See the Julia code file.
    \solend
\end{sol}
%\end{comment}
%\begin{comment}
\begin{sol}[1.3]
    The likelihood function is
    \begin{equation*}
        L(\alpha_1, \alpha_2, \alpha_3) = \prod_{i=1}^{1000}\prod_{j=0}^{J}\Pr(D_i = j)^{\mathds{1}\brc{D_i = j}} = \prod_{i=1}^{1000}\prod_{j=0}^{J}\brc{\frac{\exp(\alpha_j)}{\sum_{k=0}^{J}\exp(\alpha_k)}}^{\mathds{1}\brc{D_i = j}}.
    \end{equation*}
    \solend
\end{sol}
%\end{comment}
%\begin{comment}
\begin{sol}[1.5]
    By 1.1, for $J = 2$, we have 
    \begin{equation*}
        \frac{\Pr(D_i = 1)}{\Pr(D_i = 2)} = \frac{\exp(\alpha_1)}{\exp(\alpha_2)} = \exp(\alpha_1 - \alpha_2).
    \end{equation*}
    For $J = 3$, we have
    \begin{equation*}
        \frac{\Pr(D_i = 1)}{\Pr(D_i = 2)} = \frac{\exp(\alpha_1)}{\exp(\alpha_2)} = \exp(\alpha_1 - \alpha_2).
    \end{equation*}
    Notice that they are the same. This is called the IIA assumption. 
    One should be careful when using the multinomial logit model despite its 
    convenience.
    \solend
\end{sol}
%\end{comment}

\section{Inventory Dynamics}
Consider a firm has a warehouse to store its inventory with 
a capacity of $S$ units. The time line in each period goes as follows: 
At the beginning of period $t$, the firm observes the inventory $I_t$ and 
$\epsilon_{0t}, \epsilon_{1t}$ but not the demand $D_t$. The firm then decides 
whether to restock ($s_t = 1$) the inventory to $S$ units or not ($s_t = 0$). 
After the decision, the firm pays the storage cost $c$. Finally, the firm 
observes $D_t$ and the profit flow is realized. The profit flow is 
\begin{equation*}
    \pi(s_t, I_t, D_t) = 
    \begin{cases*}
        pD_t - cS - h(S - I_t)  & if $s_t = 1$, \\
        p\min\brc{D_t, I_t} - cI_t  & if $s_t = 0$,
    \end{cases*}
\end{equation*}  
where $I_t$ is the inventory level at the beginning of period $t$, 
$p$ is the price of the product, $c$ is the cost of holding one unit 
of inventory. $h$ is the cost of restocking with 
\begin{equation*}
    h(x) = \theta_0 + \theta_1x.
\end{equation*}
Let $\epsilon_t(s)$ be random term iid across choice and 
time following $\it{T1EV}$ distribution. We normalize it as 
\begin{equation*}
    \epsilon_t(s) = -\gamma + s\epsilon_{1t} + (1-s)\epsilon_{0t}.
\end{equation*}
Remark that $\epsilon_t$ is observed by the firm but not us.
The law of motion for the inventory level is 
\begin{equation*}
    I_{t+1} = 
    \begin{cases*}
        S - D_t & if $s_t = 1$, \\
        \max\brc{0,I_t - D_t} & if $s_t = 0$.
    \end{cases*}
\end{equation*}
\begin{enumerate}
    \item Assume that $D_t\overset{iid}{\sim}BetaBinomial(\alpha, \beta, S)$, $S=20$.
    Generate the transition array $Q$ for the inventory level $I_t$.
    \item Let the discount factor be $\rho\in (0,1)$. Write down the Bellman equation 
    for the firm's problem. 
    \item Given the parameters $(p,c,\theta_0,\theta_1,\alpha,\beta,\rho) = (1.3, 1.0, 3.0, 0.7,5.0,2.0,0.95)$,
    solve the firm's policy function using the value function 
    iteration. 
    \item Given the same parameters, solve the firm's policy function using the policy function 
    iteration. \\
    (Hint: In $T_\sigma$ step, $v_\sigma = (I-\rho Q_\sigma)^{-1}\pi_\sigma$, where $Q_\sigma$ is the transition matrix 
    given the policy and $\pi_\sigma$ is the profit flow given the policy.)
    \item The above two methods are fundamental. For this problem, with some additional assumptions, 
    it is possible to solve the model faster. Assume that the shock $\epsilon_{t+1}$ and the demand $D_t$ are 
    conditionally independent, i.e., 
    \begin{equation*}
        \Pr(I_{t+1}\mid \epsilon_t, I_t, s_t) = \Pr(I_{t+1}\mid I_t, s_t).
    \end{equation*}  
    Derive the function equation
    \begin{equation*}
        \E_\epsilon\sbrc{v(I)} = \log\brc{\sum_{s\in\set{0,1}}\exp\sbrc{\E_D\sbrc{\pi(s,I,D)} + \rho\sum_{I'}(\E_\epsilon\sbrc{v(I')})Q(I', I, s)}},
    \end{equation*}
    where $\gamma$ is the Euler-Mascheroni constant. Note that iterating on this equation forms the inner loop of the NFXP algorithm.
    \item Simulate the data $\set{(\min\brc{D_t, I_t}, I_t, s_t)}_{t=1}^{500} = \set{(d_t, I_t, s_t)}_{t=1}^{500}$. 
    Estimate the transition matrix parameters $\alpha$ and $\beta$ for the inventory level $I_t$ 
    using the simulated data.  
    \item Estimate the parameters $(p,c,\theta_0,\theta_1,\rho)$ using NFXP algorithm. 
    \item Estimate the parameters $(p,c,\theta_0,\theta_1,\rho)$ using CCP.
    \item Simulate and plot the inventory level $I_t$ against $t$ for $100$ periods for 
    different price levels $p\in\set{1.1, 1.3, 1.5}$ and storage cost $c\in\set{0.8, 1.0, 1.2}$. 
    Explain your findings.
\end{enumerate}
%\begin{comment}
\begin{sol}[1.1, 1.3, 1.4, 1.9]
    See the Julia code file.
    \solend
\end{sol}
%\end{comment}
%\begin{comment}
\begin{sol}[1.2]
    The Bellman equation is 
    \begin{equation*}
        v(I_t, \epsilon_t) = \max_{s_t}\brc{\E_D\sbrc{\pi(s_t, I_t, D_t)} + \epsilon_t(s) + \rho\sum_{I_{t+1}, \epsilon_{t+1}}v(I_{t+1}, \epsilon_{t+1})\Pr(I_{t+1}, \epsilon_{t+1}\mid I_t, s_t)}.
    \end{equation*}
    \solend
\end{sol}
%\end{comment}
%\begin{comment}
\begin{sol}[1.5]
    By the conditional independence assumption, we have 
    \begin{equation*}
        \begin{aligned}
            \E_{(\epsilon_{t+1}, I_{t+1})}\sbrc{v(I_{t+1}, \epsilon_{t+1})}
            &= \sum_{I_{t+1}, \epsilon_{t+1}} v(I_{t+1}, \epsilon_{t+1}) \Pr(I_{t+1}, \epsilon_{t+1}\mid I_t, s_t) \\
            &= \sum_{I_{t+1}}\sum_{\epsilon_{t+1}} v(I_{t+1}, \epsilon_{t+1})\Pr(I_{t+1}\mid \epsilon_{t+1}, I_t, s_t)\Pr(\epsilon_{t+1}\mid I_t, s_t)\\
            &= \sum_{I_{t+1}}\sum_{\epsilon_{t+1}} v(I_{t+1}, \epsilon_{t+1})\Pr(I_{t+1}\mid I_t, s_t)\Pr(\epsilon_{t+1}\mid I_t, s_t)\\
            &= \sum_{I_{t+1}}\E_{\epsilon_{t+1}}\sbrc{v(I_{t+1}, \epsilon_{t+1})}\Pr(I_{t+1}\mid I_t, s_t).
        \end{aligned}
    \end{equation*}
    Then 
    \begin{equation*}
        \begin{aligned}
            \E_\epsilon\sbrc{v(I,\epsilon)} 
            &= \int_\epsilon \max_s\brc{\E_D\sbrc{\pi(s,I,D)} + \epsilon(s) + \rho\sum_{I'}\E_\epsilon\sbrc{v(I',\epsilon')}P(I'\mid I, s)}dF(\epsilon)\\ 
            &= \log\brc{\sum_{s\in\set{0,1}}\exp\sbrc{\E_D\sbrc{\pi(s,I,D)} + \rho\sum_{I'}\E_\epsilon\sbrc{v(I',\epsilon')}Q(I', I, s)}}
        \end{aligned}
    \end{equation*}
    by the $\it{T1EV}$ assumption.
    \solend
\end{sol}
%\end{comment}
%\begin{comment}
\begin{sol}[1.6]
    We derived the likelihood function here. 
    \begin{equation*}
        \mathcal{L} = \prod_{t=1}^{500}\Pr(\min\brc{D_t, I_t} = d_t\mid I_t) = \prod_{t=1}^{500}f(d_t)^{\mathds{1}\brc{d_t < I_t}}(1-F(I_t))^{\mathds{1}\brc{d_t = I_t}},
    \end{equation*}
    where $f$ is the probability mass function of the Beta-Binomial distribution and $F$ is the cumulative distribution function.
    \solend
\end{sol}
%\end{comment}
%\begin{comment}
\begin{sol}[1.7]
    Once we obtain $\E_\epsilon\sbrc{v(I,\epsilon)}$ for the given parameters, 
    we can calculate the associated likelihood via 
    \begin{equation*}
        \begin{aligned}
            \mathcal{L} &= \prod_{t=1}^{500}\Pr(\E_D\sbrc{\pi(0,I_t,D)} + \epsilon_t(0) + \rho\sum_{I_{t+1}}\E_\epsilon\sbrc{v(I_{t+1}, \epsilon_{t+1})}Q(I_{t+1}, I_t, 0) \\
            &\qquad > \E_D\sbrc{\pi(1,I_t,D)} + \epsilon_t(1) + \rho\sum_{I_{t+1}}\E_\epsilon\sbrc{v(I_{t+1})}Q(I_{t+1}, I_t, 1))^{\mathds{1}\brc{s_t = 0}} \\
            &\quad \times \Pr(\E_D\sbrc{\pi(1,I_t,D)} + \epsilon_t(1) + \rho\sum_{I_{t+1}}\E_\epsilon\sbrc{v(I_{t+1}, \epsilon_{t+1})}Q(I_{t+1}, I_t, 1) \\
            &\qquad > \E_D\sbrc{\pi(0,I_t,D)} + \epsilon_t(0) + \rho\sum_{I_{t+1}}\E_\epsilon\sbrc{v(I_{t+1})}Q(I_{t+1}, I_t, 0))^{\mathds{1}\brc{s_t = 1}}\\
            &= \prod_{t=1}^{500}\prod_{s} \brc{\frac{\E_D\sbrc{\pi(s,I_t,D)} + \rho\sum_{I_{t+1}}\E_\epsilon\sbrc{v(I_{t+1}, \epsilon_{t+1})}Q(I_{t+1}, I_t, s)}{\sum_{s'} \exp\sbrc{\E_D\sbrc{\pi(s',I_t,D)} + \rho\sum_{I_{t+1}}\E_\epsilon\sbrc{v(I_{t+1}, \epsilon_{t+1})}Q(I_{t+1}, I_t, s')}}}^{\mathds{1}\brc{s_t = s}}.
        \end{aligned}
    \end{equation*}
    \solend
\end{sol}

\end{document}
