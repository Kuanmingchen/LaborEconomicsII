\documentclass[12pt]{article}
\input{prefix.tex}%
\input{symbols.tex}

\title{
    Labor Economics Homework 3 \\
    \large Due: \\
    }

\author{}

\date{}

\begin{document}
\setstretch{1.2}

\maketitle
\vspace{-2cm}

\section{Estimation of McCall's Job Search Model} 
Recall our settings from homework 1. An unemployed worker solves the 
following Bellman equation:
\begin{equation*}
    v(w) = \max_{s} sw + (1-s)\sbrc{\underbar{c} + \rho \int v(w') F(dw')},
\end{equation*}
where $F$ is the CDF of the wage offer distribution $Beta(\alpha, \beta)$.
\begin{enumerate}
    \item Use the parameters in homework 1.3 to simulate the data 
    for one period, i.e., $\set{(w_i, s_i)}_{i=1}^{1000}$. 
    Note that we do not observe the wage if $s_i = 0$.
    \item Compute the mean wage among those who are employed. Is the mean wage you computed
    close with the mean wage in the distribution of wage offers? \\
    (Hint: $X\overset{iid}{\sim}Beta(\alpha, \beta)$ implies $\Ex{X} = \frac{\alpha}{\alpha + \beta}$.)
    \item Estimate the wage distribution parameters $\alpha$ and $\beta$ by MLE. \\
    (Hint: The observed wage follows a truncated beta distribution.) 
    \item We do know $\underbar{c} = 0.1$ since it is the unemployment benefit. 
    Estimate $\rho$.
\end{enumerate}

%\begin{comment}
\begin{sol}[1.1, 1.2, 1.4]
    See the Julia code file.
\end{sol} 
%\end{comment}

%\begin{comment}
\begin{sol}[1.3]
    Notice that the reservation wage $\Bar{w}$ is the threshold. The MLE estimator is 
    \begin{equation*}
        \argmax_{\alpha, \beta, \Bar{w}} \sum_{i=1}^{1000} s_i\log \frac{f(w_i;\alpha, \beta)}{1 - F(\Bar{w};\alpha, \beta)} + (1-s_i)\log F(\Bar{w};\alpha, \beta),
    \end{equation*}
    where $f$ is the PDF of the beta distribution and $F$ is the CDF. Implementation is in the Julia code file.
\end{sol}
%\end{comment}


\section{Estimation of Crusoe's Coconuts Model}
Recall our settings from homework 2. Crusoe solves the following Bellman equation: 
\begin{equation*}
    v(y) = \max_{c} u(c) + \beta \int v(y') F(dy'\mid y, c),
\end{equation*}
where $F(dy'\mid y, c)$ is the transition probability from $y$ to $y'$ given $c$, 
with the law of motion $y' = zf(y - c)$, $f(k) = k^\alpha$ and $\log z\overset{iid}{\sim} N(0, \sigma^2)$.
\begin{enumerate}
    \item Use the parameters in homework 2.3 to simulate the time series data 
    $\set{(y_t, k_t, c_t)}_{t=1}^{200}$. 
    \item Estimate the production process parameters $\alpha$, $\mu$ and $\sigma$.  
    \item Estimate $\beta$ and $\gamma$ by SMM. What 
    moment conditions would you use? 
\end{enumerate}
%\begin{comment}
\begin{sol}[1.1, 1.2]
    See the Julia code file.
\end{sol} 
%\end{comment}

%\begin{comment}
\begin{sol}[1.3]
    I would use 
    \begin{equation*}
        \Ex{c_t - \hat{c}_t|y_t} = 0,
    \end{equation*}
    but you may use other moment conditions.
\end{sol}
%\end{comment}

\end{document}