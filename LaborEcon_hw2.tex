\documentclass[12pt]{article}
\input{prefix.tex}%
\input{symbols.tex}

\title{
    Labor Economics Homework 2 \\
    \large Due: \\
    }

\author{}

\date{}

\begin{document}
\setstretch{1.2}

\maketitle
\vspace{-2cm}

\section{Robinson Crusoe's Coconuts Revisited}
Recall our settings from homework 1. This time, 
Robinson Crusoe's utility is a CRRA function 
\begin{equation*}
    u(c; \gamma) = \lim_{\gamma'\to\gamma}\frac{c^{1-\gamma'}}{1-\gamma'}
\end{equation*}
with $\gamma\geq 0$. At the beginning of each period 
$t$, Crusoe owns $y_t$ coconuts. He 
can either consume them or plant them. If he plants 
$k_{t+1}$ coconuts, he will obtain $y_{t+1} = 
z_{t+1}f(k_{t+1})$ coconuts in the next period, where 
$z_{t+1}\overset{iid}{\sim} Lognormal(\mu,\sigma^2)$. 
The coconuts rot after each period, thus $k_t$ 
is fully depreciated. We assume that $f(k) = 
k^\alpha$ with $\alpha \in (0,1)$. With discount 
factor $\beta\in(0,1)$, Crusoe's problem is 
\begin{equation*}
    \max_{\{c_t,k_{t+1}\}_{t=0}^\infty} \sum_{t=0}^\infty 
    \E_0\sbrc{\beta^t u(c_t)} \quad \text{s.t.} \quad 
    c_t + k_{t+1} = y_t \quad \text{and} \quad y_{t+1} = 
    z_{t+1}f(k_{t+1})
\end{equation*} 
\begin{enumerate}
    \item Write down the Bellman equation and define the Bellman operator. 
    \item Show that the Bellman operator is a contraction mapping.\\
    (Hint: Verify the Blackwell's sufficient conditions.)
    \item Given $\alpha = 0.8$, $\beta = 0.96$, $\gamma = 1$, $\mu = 0.0$, $\sigma = 0.3$. 
    Solve the model by value function iteration for $y\in(0,10)$.\\
    (Hint: You may use the linear interpolation and monte carlo simulation to 
    approximate the expectation of $v$.)
    \item Given the same parameters, solve the model by policy function 
    iteration. 
    \item Given the same parameters, solve the model by envelope condition 
    methods. Use both the exogenous grids and the endogenous grids.  
    Compare the running time of the four methods using macro 
    \texttt{@benchmark} in Julia. 
    \item The model has a famous closed-form solution with $\gamma = 1$:
    \begin{equation*}
        c^*(y) = (1-\alpha\beta)y.
    \end{equation*}
    Plot the consumption policies for the analytical and numerical solutions. 
    Do your numerical solutions fit the analytical one well?
    \item Solve the model for $\sigma = 0.1, 0.015, 0.2, 0.25$ and $0.3$ with 
    $\gamma = 1.5$. Plot the consumption policies for different $\sigma$. 
    Explain your findings. 
\end{enumerate}

\begin{sol}[1.1]
    The Bellman equation is 
    \begin{equation*}
        v(y) = \max_{c} u(c) + \beta \int v(zf(y-c)) F(dz),
    \end{equation*}
    where $F$ stands for the CDF of $z$. The Bellman operator is 
    \begin{equation*}
        (Tv)(y) = \max_{c} u(c) + \beta \int v(zf(y-c)) F(dz).
    \end{equation*}
    \solend
\end{sol}

\begin{sol}[1.2]
    We are going to check the conditions in Blackwell's Theorem are 
    satisfied. First, if $v\leq w$, $v,w\in B(X)$, let $c_v$ and 
    $c_w$ be the optimal in Bellman equations for $v$ and $w$, respectively. 
    Then 
    \begin{equation*}
        \begin{aligned}
            Tv(y) 
            &= u(c_v) + \beta\int v(zf(y-c_v))\phi(dz)\\
            &\leq u(c_v) + \beta\int w(zf(y-c_v))\phi(dz)\\
            &\leq u(c_w) + \beta\int w(zf(y-c_w))\phi(dz) = Tw(y).
        \end{aligned}
    \end{equation*} 
    Thus the monotonicity is satisfied. Second, for any $a\in\R_+$, 
    \begin{equation*}
        \begin{aligned}
            T(v+a)(y) 
            &= \max_c u(c) + \beta\int (v+a)(zf(y-c))\phi(dz)\\
            &= \max_c u(c) + \beta\int v(zf(y-c))+a \phi(dz)
            = Tv(y) + \beta a.
        \end{aligned}
    \end{equation*}
    Note that here by $(v+a)(y)$ we mean $v(y) + a$. This completes 
    the proof.
    \solend
\end{sol}

\end{document}